\documentclass{article}

\usepackage{siunitx} % Provides the \SI{}{} and \si{} command for typesetting SI units
\usepackage{graphicx} % Required for the inclusion of images
\usepackage{amsmath} % Required for some math elements 
\usepackage[export]{adjustbox} % loads also graphicx
\usepackage{listings}
\usepackage{matlab-prettifier}
\usepackage{float}

\usepackage{titlesec}
\usepackage{caption}
\usepackage{subcaption}


\usepackage{xcolor}

\DeclareCaptionFont{white}{\color{white}}
\DeclareCaptionFormat{listing}{%
  \parbox{\textwidth}{\colorbox{gray}{\parbox{\textwidth}{#1#2#3}}\vskip-4pt}}
\captionsetup[lstlisting]{format=listing,labelfont=white,textfont=white}
\lstset{frame=lrb,xleftmargin=\fboxsep,xrightmargin=-\fboxsep}
\titleformat{\section}[runin]
  {\normalfont\Large\bfseries}{\thesection}{1em}{}
\titleformat{\subsection}[runin]
  {\normalfont\large\bfseries}{\thesubsection}{1em}{}


\setlength\parindent{0pt} % Removes all indentation from paragraphs

\renewcommand{\labelenumi}{\alph{enumi}.} % Make numbering in the enumerate environment by letter rather than number (e.g. section 6)

%\usepackage{times} % Uncomment to use the Times New Roman font

%----------------------------------------------------------------------------------------
%	DOCUMENT INFORMATION
%----------------------------------------------------------------------------------------

\title{AMATH 353: Homework 3 \\Due April, 11 2018 \\ ID: 1064712} % Title

\author{Trent \textsc{Yarosevich}} % Author name

\date{\today} % Date for the report

\begin{document}
\maketitle % Insert the title, author and date
\setlength\parindent{1cm}

\begin{center}
\begin{tabular}{l r}
%Date Performed: December 1, 2017 \\ % Date the experiment was performed
Instructor: Jeremy Upsal % Instructor/supervisor
\end{tabular}
\end{center}

% If you wish to include an abstract, uncomment the lines below
% \begin{abstract}
% Abstract text
% \end{abstract}

%----------------------------------------------------------------------------------------
%	SECTION 1
%----------------------------------------------------------------------------------------
\section*{Part 1.)} 
\subsection*{a.)}
By $u(x, t)=f(x-ct)$ and $z=x-ct$ the PDE (Klein-Gordon) \\ $u_{tt} = au_{xx} - bu$ becomes:
\begin{equation}
c^2f''(z) = af''(z)+bf(z)
\end{equation}     
\subsection*{b.)}
Rewriting the ODE as $(c^2 - a)f''(z) + bf(z) = 0$ we arrive at a trivial solution when $c^2 = a$, or when $c = \pm\sqrt{a}$. This value of c results in the second order derivative being zero, yielding the trivial solution $bf(z) = 0$ or $f(z) = 0$.\\
\\
Accordingly, we get non-trivial solutions both when $c^2 > a$ and when $c^2 < a$.
\subsection*{c.)}
If we let $A^2 = \frac{b}{c^2 -a}$ we then have the equation
\begin{equation}
f''(z) + A^2f(z) = 0
\end{equation}
which we then solve using the characteristic polynomial for both cases of non-trivial solutions. When $c^2 > a$ it follows that $A^2$ will be positive, resulting in:
\begin{equation}
r^2 + A^2 = 0
\end{equation}
and by quadratic formula we find the roots
\begin{equation}
r = \pm Ai
\end{equation}
which gives us a solution 
\begin{equation}
f(z) = C_1cos(Ax) + C_2sin(Ax)
\end{equation}
This then results in an oscillatory solution for values of $c^2 > a$
\begin{equation}
u(x, t) = C_1cos(\sqrt{\frac{b}{c^2 - a}}(x - ct)) + C_2sin(\sqrt{\frac{b}{c^2 - a}}(x - ct))
\end{equation}
\subsection*{d.)}
For values of $c^2<a$ we take the same approach, but $A^2$ is now negative, resulting in the following with the characteristic polynomial:
\begin{equation}
\begin{aligned}
r^2 - A^2 = 0\\
(r - A)(r + A) = 0\\
r = \pm A
\end{aligned}
\end{equation}
resulting in a solution of
\begin{equation}
f(z) = C_1e^{Az} + C_2e^{-Az}
\end{equation}
If we then let $C_1 = 0$ we arrive at a solution that decays as $x \to \infty$ for values of $c^2 < a$. Substituting back in for $z$:
\begin{equation}
u(x,t) = C_2e^{-\sqrt{\frac{b}{c^2 - a}}(x-ct)}
\end{equation}
\section*{Part 2.)}
Letting $u(x, t) = f(x - ct)$ and $z = x-ct$ we arrive at the ODE
\begin{equation}
c^2f''(z) - f''(z) + sin(f(z)) = 0
\end{equation}
Using the shorthand $f$ in place of $f(z)$, we multiply by $f'$ and arrive at 
\begin{equation}
(c^2 -1)f''f' + sin(f)f' = 0
\end{equation}
and then integrate to get
\begin{equation}
\frac{1}{2}(c^2 - 1)(f')^2 - cos(f) = a
\end{equation}
Because we are treating $f(z) \to \pi$ and $f'(z) \to 0$ as $z \to \infty$ we can find the constant of integration a:
\begin{equation}
\begin{aligned}
\frac{1}{2}(c^2 - 1)(0)^2 - cos(\pi) = a\\
a = 1
\end{aligned}
\end{equation}
We then have
\begin{equation}
\frac{1}{2}(c^2 -1)(f')^2 = 1+ cos(f) 
\end{equation}
then by half-angle formula
\begin{equation}
\frac{1}{4}(c^2 -1)(f')^2 = cos^2(f/2) 
\end{equation}
With some manipulation this becomes
\begin{equation}
\begin{aligned}
\frac{df}{cos(\frac{f}{2})} = \frac{2}{\sqrt{c^2-1}}dz\\
sec(\frac{f}{2}) = \frac{2}{\sqrt{c^2-1}}dz
\end{aligned}
\end{equation}
\noindent
Integration using the method of multiplying $sec$ by $\frac{sec}{tan}$ yields
\begin{equation}
\begin{aligned}
2ln(sec(\frac{f}{2}) + tan(\frac{f}{2})) = \frac{2}{\sqrt{c^2-1}} z + E\\
sec(\frac{f}{2}) + tan(\frac{f}{2}) = e^{\frac{1}{\sqrt{c^2-1}}} z + E
\end{aligned}
\end{equation}
By an obscure trig identity we get
\begin{equation}
1 + \frac{2}{cot(\frac{f}{4}) -1} = e^{\frac{1}{\sqrt{c^2-1}}z}
\end{equation}
By rearrangement and taking the inverse of both sides, we get a function of tangent. By then taking the arctan, and assuming the constant of integration is 0, we arrive at the following equation, which was verified to solve the PDE using Wolfram:
\begin{equation}
f = 4arctan(\frac{1}{\frac{2}{e^{\frac{z}{\sqrt{c^2-1}}+1}}})
\end{equation}
Finally, substituting back in for $z$ and returning the constant of integration $E$ we get:
\begin{equation}
f = 4arctan(\frac{1}{\frac{2}{e^{\frac{x-ct + E}{\sqrt{c^2-1}}+1}}})
\end{equation}
I'm not certain, but I believe the physical interpretation of this, in the pendulum schema, is that after the wavefront passes, the pendula have rotated halfway and are all on the opposite side from where they started before the wavefront passed them.
\section*{Part 3}
By substituting $u$ into the PDE we get the following equation and expansion:
\begin{equation}
\begin{aligned}
u_t +uu_x + u_{xxx} = 0\\
v_t + w_t + (v + w)(v_x + w_x) + v_{xxx} + w_{xxx} = 0\\
v_t + w_t + vv_x + vw_x + wv_x +ww_x + v_{xxx} + w_{xxx} = 0\\
\end{aligned}
\end{equation}
After cancelling of terms we arrive at:
\begin{equation}
\begin{aligned}
wv_x + vw_x = 0\\
\frac{d}{dx}(wv) = 0
\end{aligned}
\end{equation}
Integration would show that in order for this solution to satisfy the PDE, $wv$ cannot be dependent upon $x$, and could only have constants and $t$ terms. Thus this solution models an interesting situation in which there is no spatial interaction between the two wave solutions.
\end{document}