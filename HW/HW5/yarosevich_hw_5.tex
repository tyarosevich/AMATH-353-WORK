\documentclass{article}

\usepackage{siunitx} % Provides the \SI{}{} and \si{} command for typesetting SI units
\usepackage{graphicx} % Required for the inclusion of images
\usepackage{amsmath} % Required for some math elements 
\usepackage[export]{adjustbox} % loads also graphicx
\usepackage{listings}
\usepackage{matlab-prettifier}
\usepackage{float}

\usepackage{titlesec}
\usepackage{caption}
\usepackage{subcaption}


\usepackage{xcolor}

\DeclareCaptionFont{white}{\color{white}}
\DeclareCaptionFormat{listing}{%
  \parbox{\textwidth}{\colorbox{gray}{\parbox{\textwidth}{#1#2#3}}\vskip-4pt}}
\captionsetup[lstlisting]{format=listing,labelfont=white,textfont=white}
\lstset{frame=lrb,xleftmargin=\fboxsep,xrightmargin=-\fboxsep}
\titleformat{\section}[runin]
  {\normalfont\Large\bfseries}{\thesection}{1em}{}
\titleformat{\subsection}[runin]
  {\normalfont\large\bfseries}{\thesubsection}{1em}{}


\setlength\parindent{0pt} % Removes all indentation from paragraphs

\renewcommand{\labelenumi}{\alph{enumi}.} % Make numbering in the enumerate environment by letter rather than number (e.g. section 6)

%\usepackage{times} % Uncomment to use the Times New Roman font

%----------------------------------------------------------------------------------------
%	DOCUMENT INFORMATION
%----------------------------------------------------------------------------------------

\title{AMATH 353: Homework 5 \\Due April, 18 2018 \\ ID: 1064712} % Title

\author{Trent \textsc{Yarosevich}} % Author name

\date{\today} % Date for the report

\begin{document}
\maketitle % Insert the title, author and date
\setlength\parindent{1cm}

\begin{center}
\begin{tabular}{l r}
%Date Performed: December 1, 2017 \\ % Date the experiment was performed
Instructor: Jeremy Upsal % Instructor/supervisor
\end{tabular}
\end{center}

% If you wish to include an abstract, uncomment the lines below
% \begin{abstract}
% Abstract text
% \end{abstract}

%----------------------------------------------------------------------------------------
%	SECTION 1
%----------------------------------------------------------------------------------------
\section*{Part 1.)} 
\subsection*{a.)}
If $(u,w) = (0, 0)$ then we have:
\[
  \begin{cases}
               0 = 0 + 0(0 - a)(1-0) + 0\\
               0 = \epsilon 0
            \end{cases}
\]
Thus in the trivial case, $a$ and $\epsilon$ will always be a product with zero, and the equations will be satisfied.
\subsection*{b.)}
Beginning with the first equation $u_t = u_{xx} + u(u - a)(1-u) + w$ and substituting $u = \alpha\hat{u}$ and $w = \alpha\hat{w}$ (note I will expand the equation and move all terms to the LHS before substitution):
\begin{equation}
\begin{aligned}
u_t - u_{xx} - u^2 + u^3 + au - au^2 - w = 0\\
\alpha\hat{u}_t - \alpha\hat{u}_{xx} - \alpha^2\hat{u}^2 + \alpha^3\hat{u}^3 + \alpha a\hat{u} - \alpha^2 a\hat{u}^2 - \hat{w} = 0\\
\alpha (\hat{u}_t - \hat{u}_{xx} - \alpha\hat{u}^2 + \alpha^2\hat{u}^3 +  a\hat{u} - \alpha a\hat{u}^2 - \hat{w}) = 0\\
\end{aligned}
\end{equation}
We then divide out the $\alpha$ that was pulled out from all the LHS terms, and cancel all remaining terms that have an $\alpha^n$ constant because it is very small, leaving
\begin{equation}
\hat{u}_t - \hat{u}_{xx} + a\hat{u} - \hat{w} = 0
\end{equation}
into which we can substitute $u$ and $w$, obtaining the linear form of the equation.\\
\\
For the second equation we do the same, though it is very straightforward:
\begin{equation}
\begin{aligned}
\alpha\hat{w}_t - \alpha\epsilon\hat{u} = 0\\
\alpha(\hat{w}_t - \epsilon\hat{u}) = 0\\
\hat{w}_t - \epsilon\hat{u} = 0
\end{aligned}
\end{equation}
We then substitute $u$ and $w$ here as well.
\subsection*{c.)}
\[
\begin{bmatrix}
(-i\omega e^{ikx - i\omega t} - (ik)^2e^{ikx - i\omega t} + e^{ikx - i\omega t}) & -1\\
-1 & -i\omega
\end{bmatrix}
\begin{bmatrix}
U\\
W
\end{bmatrix}
= 0
\]
\subsection*{d.)}
To begin, we divide out the $e^{ikx - i\omega t}$ terms in the first equation and compute the $i^2$, yielding the following matrix:
\[
A =
\begin{bmatrix}
(-i\omega e + k^2 + 1) & -1\\
-1 & -i\omega
\end{bmatrix}
\]
Taking the determinant of this matrix and expanding:
\begin{equation}
\begin{aligned}
(-i\omega + k^2 + 1)(-i\omega) - (-1)(-1) = 0\\
i^2\omega^2 - k^2i\omega - i\omega -1 = 0\\
-\omega^2 - k^2i\omega -i\omega -1 = 0\\
\omega^2 + k^2i\omega + i\omega + 1 = 0\\
\omega^2 + (ik^2 + i)\omega + 1 =0
\end{aligned}
\end{equation}
We then apply the quadratic formula and arrive at:
\begin{equation}
\omega = \frac{-(ik^2 + i) \pm \sqrt{(ik^2 + i)^2 -4}}{2}
\end{equation}
We now simplify the numerator to see if it can be arranged such that, eventually, $iw$ will be equal to some real quantity. Please note that for the sake of typesetting, the RHS denominator will be moved to the LHS while the simplification of the numerator is shown.
\begin{equation}
\begin{aligned}
2\omega = -(ik^2 + i) \pm \sqrt{(ik^2 + i)^2 -4}\\
2\omega = -(ik^2 + i) \pm \sqrt{i^2(k^2 + 1)^2 -4}\\
2\omega = -(ik^2 + i) \pm \sqrt{(-1)(k^2 + 1)^2 -4}\\
2\omega = -(ik^2 + i) \pm \sqrt{(-1)((k^2 + 1)^2 +4)}\\
2\omega = -(ik^2 + i) \pm \sqrt{-1}\sqrt{(k^2 + 1)^2 +4}\\
2\omega = -(ik^2 + i) \pm i\sqrt{(k^2 + 1)^2 +4}\\
2\omega = -i(k^2 + 1) \pm i\sqrt{(k^2 + 1)^2 +4}\\
\end{aligned}
\end{equation}
We then multiply across by $i$ and simplify, yielding $iw$ equal to a real quantity:
\begin{equation}
\begin{aligned}
2i\omega = -i^2(k^2 + 1) \pm i^2\sqrt{(k^2 + 1)^2 +4}\\
2i\omega = (k^2 + 1) \pm \sqrt{(k^2 + 1)^2 +4}\\
\text{giving the two equations that satisfy $det(A)$}\\
i\omega_+ = \frac{(k^2 + 1) + \sqrt{(k^2 + 1)^2 +4}}{2}\\
\text{and}\\
i\omega_- = \frac{(k^2 + 1) - \sqrt{(k^2 + 1)^2 +4}}{2}\\
\end{aligned}
\end{equation}
\subsection*{e.)}
If we split $u(x, t)$ and $w(x,t)$ into their real and imaginary parts and substitute in the results from part d.) we get the following equations:
\begin{equation}
\begin{aligned}
u(x, t) = Ue^{-\frac{(k^2 + 1) \pm \sqrt{(k^2 + 1)^2 +4}}{2}t}e^{ikx}\\
\text{and}\\
w(x, t) = We^{-\frac{(k^2 + 1) \pm \sqrt{(k^2 + 1)^2 +4}}{2}t}e^{ikx}\\
\end{aligned}
\end{equation}
Since the equations differ only in terms of their constant, we need only study one in order to determined if both will grow or decay. The question is whether or not the constant in the imaginary part is greater than zero. If it is, the exponent will be negative and the solution will decay; if it is negative, it will grow. The denominator in $\omega$ is irrelevant in this regard, thus we must investigate the two inequalities:
\begin{equation}
(k^2 + 1) + \sqrt{(k^2 + 1)^2 +4} > 0
\end{equation}
\begin{equation}
(k^2 + 1) - \sqrt{(k^2 + 1)^2 +4} > 0
\end{equation}
In equation 9, it is self-evident that this inequality is always true for any real $k$, since all $k$ values are squared and no subtraction takes place. This in turn means that for $\omega_+$ both $u$ and $w$ will decay with all real $k$ values.
Similarly with equation 10, we have the inequality
\begin{equation}
\begin{aligned}
(k^2 + 1) > \sqrt{(k^2 + 1)^2 +4}\\
(k^2 + 1)^2 > (k^2 +1)^2 + 4
\end{aligned}
\end{equation}
In this form of $\omega_-$ it is also self-evident that the inequality is never satisfied for any real $k$. In both cases, equations 9 and 10, this is because the polynomials have no roots. In this case it means that $\omega_-$ is never positive, and thus when used to find solutions of $u$ and $w$ they will always decay.
\\
\\
To summarize, solutions for both $u$ and $w$ will oscillate because of the $e^{ikx}$ term; solutions using $\omega_+$ will always decay; and solutions using $\omega_-$ will always grow.
\subsection*{f.)}
The dispersion relation $c_p(k)$ for both $u(x,t)$ and $w(x, t)$ is 
\begin{equation}
c_p(k) = \frac{(k^2 + 1) \pm \sqrt{(k^2 + 1)^2 +4}}{2ik}
\end{equation}
Because $c_p(k)$ is complex (the denominator is $2ik$), this system of equations is not dispersive. 
\section*{Part 2.)}
\subsection*{a.)}
Done worked through it.
\subsection*{b.)}
As with the example derivation we let $S$ represent an imaginary segment of string that we are considering to be on the $x$ axis, lying between $x$ and $x + \Delta x$, with $\Delta x >0$ and very small. We then utilize Newton's second law of motion, assuming that the string is moving up and down only a very small amount, such that the motion $u(x, t)$ can be taken to be perpendicular to the $x$ axis, and as such the acceleration and net force are also acting perpendicular to the $x$ axis.
\begin{equation}
\text{Mass of $S$)(Acceleration of $S$) = Net force acting on $S$}
\end{equation}
So we now need to find these terms. Acceleration is easy, as it is just the second derivative in time, $u_{tt}(x, t)$. \\
\indent The mass of the segment $S$ is the density $\rho(x)$ times the length of $S$. The density of $S$ is thus given by $\rho (s) \approx \rho (x)$ for $x < s < x + \Delta x$.
\begin{equation}
\text{Mass} (S) = \int_{x}^{x + \Delta x} \rho (s) \sqrt{1 + u_x(s,t))^2}ds
\end{equation}
with $u_x$ assumed to be very small because the vibrations on the string are very small, thus $u_x(s,t)^2$ is approximately $0$, giving
\begin{equation}
\begin{aligned}
\text{Mass} (S) = \int_{x}^{x + \Delta x} \rho (s) \sqrt{1}ds\\
\text{Mass} (S) = \int_{x}^{x + \Delta x} \rho (s)ds
\end{aligned}
\end{equation}
bla
\end{document}