\documentclass{article}

\usepackage{siunitx} % Provides the \SI{}{} and \si{} command for typesetting SI units
\usepackage{graphicx} % Required for the inclusion of images
\usepackage{amsmath} % Required for some math elements 
\usepackage[export]{adjustbox} % loads also graphicx
\usepackage{listings}
\usepackage{matlab-prettifier}
\usepackage{float}
\usepackage[most]{tcolorbox}
\usepackage{amsfonts}
\usepackage{color}
\usepackage{titlesec}
\usepackage{caption}
\usepackage{subcaption}

\newcommand{\R}{\mathbb{R}}

\usepackage{xcolor}

\DeclareCaptionFont{white}{\color{white}}
\DeclareCaptionFormat{listing}{%
  \parbox{\textwidth}{\colorbox{gray}{\parbox{\textwidth}{#1#2#3}}\vskip-4pt}}
\captionsetup[lstlisting]{format=listing,labelfont=white,textfont=white}
\lstset{frame=lrb,xleftmargin=\fboxsep,xrightmargin=-\fboxsep}
\titleformat{\section}[runin]
  {\normalfont\Large\bfseries}{\thesection}{1em}{}
\titleformat{\subsection}[runin]
  {\normalfont\large\bfseries}{\thesubsection}{1em}{}


\setlength\parindent{0pt} % Removes all indentation from paragraphs

\renewcommand{\labelenumi}{\alph{enumi}.} % Make numbering in the enumerate environment by letter rather than number (e.g. section 6)

%\usepackage{times} % Uncomment to use the Times New Roman font

%----------------------------------------------------------------------------------------
%	DOCUMENT INFORMATION
%----------------------------------------------------------------------------------------

\title{AMATH 353: Homework 12 \\Due May, 18 2018 \\ ID: 1064712} % Title

\author{Trent \textsc{Yarosevich}} % Author name

\date{\today} % Date for the report

\begin{document}
\maketitle % Insert the title, author and date
\setlength\parindent{1cm}

\begin{center}
\begin{tabular}{l r}
%Date Performed: December 1, 2017 \\ % Date the experiment was performed
Instructor: Jeremy Upsal % Instructor/supervisor
\end{tabular}
\end{center}

% If you wish to include an abstract, uncomment the lines below
% \begin{abstract}
% Abstract text
% \end{abstract}

%----------------------------------------------------------------------------------------
%	SECTION 1
%----------------------------------------------------------------------------------------
\section*{Part 1}
Assuming $u(x, t) = u(x(t), t)$ , by the chain rule we have $\frac{d}{dt}u = u_t + u_x\frac{dx}{dt}$. Given that we are solving $u_t + 2u_x = 0$, if we assume $\frac{dx}{dt} = 2$ we get the following:
\begin{equation}
\frac{d}{dt}(u(x(t), t)) = u_t + 2u_x = 0
\end{equation}
This gives us the two ODEs:
\begin{tcolorbox}[minipage,colback=white,arc=0pt,outer arc=0pt]
\begin{equation}
\begin{aligned}
\frac{dx}{dt} = 2\\
\frac{du}{dt} = 0
\end{aligned}
\end{equation}
\end{tcolorbox}
Solving the first ODE by separation of variables, we get the following equation for the characteristic curves, which are shown in the plot below:
\begin{equation}
x(t) = 2t + x_0
\end{equation}
\begin{figure}[H]
  \centering
    \includegraphics[width=\textwidth]{hw_12_plot1.pdf}
    \caption{$t = \frac{x-x_0}{2}$}
\end{figure}
Solving the second ODE we simply get a constant:
\begin{equation}
\begin{aligned}
\int \frac{du}{dt} = \int 0dt\\
u(x(t), t) = A
\end{aligned}
\end{equation}
Making use of the initial condition, $u(x, 0) = e^{-x^2}$ we have the following:
\begin{equation}
\begin{aligned}
u(x(t), 0) = u(x_0, 0) = u_0(x_0) = A\\
u_0(x_0) = e^{-x_0^2}\\
A = e^{-x_0^2}\\
\end{aligned}
\end{equation}
\begin{tcolorbox}[minipage,colback=white,arc=0pt,outer arc=0pt]
\begin{equation}
u(x(t), t) = e^{-x_0^2}
\end{equation}
\end{tcolorbox}
This means that along any given characteristic line $u(x, t) = u(x(t), t)$ we have $u$ constant at a value determined by the initial value of that particular characteristic.\\
\\
Now using an example of a point $(3,4)$ we plug it into the characteristic curve and determine it's $x_0$ value, then determine the value of $u_0$ at that point, and thus $u$ along that entire characteristic curve:
\begin{equation}
\begin{aligned}
x_0 = x - 2t\\
x_0 = 3 - 4(4) = -5\\
\end{aligned}
\end{equation}
\begin{tcolorbox}[minipage,colback=white,arc=0pt,outer arc=0pt]
\begin{equation}
u_0(-5) = e^{-(-5)^2} = e^{-25}
\end{equation}
\end{tcolorbox}
We can arrive at the same value generally by plugging the $x_0$ equation into the equation derived above for $u(x(t), t)$, then plugging $(3,4)$ into that:
\begin{equation}
\begin{aligned}
x_0 = x - 2t\\
u(x(t), t) = e^{-x_0^2}\\
u(x, t) = e^{-(x - 2t)^2}\\
\end{aligned}
\end{equation}
\begin{tcolorbox}[minipage,colback=white,arc=0pt,outer arc=0pt]
\begin{equation}
e^{-(3 - 2(4)^2} = e^{-25}
\end{equation}
\end{tcolorbox}
Note the results of (8) and (10) are the same.
\section*{Part 2}
\subsection*{a.)}
The characteristic curves and ODEs for this question are arrived at in the same manner as equations (1) - (4) above, albeit restricted to $x \geq 0$:
\begin{tcolorbox}[minipage,colback=white,arc=0pt,outer arc=0pt]
\begin{equation}
\begin{aligned}
\frac{dx}{dt} = 2\\
x(t) = 2t + x_0\\
\frac{du}{dt} = 0\\
u(x(t), t) = A
\end{aligned}
\end{equation}
\end{tcolorbox}
Note that $x_0 = x - 2t$, and so with the restriction that $x > 2t$, we can assume that $x_0$ will be positive, and thus will be defined by the initial value $u(x, 0) = 0$. We can then use this to solve the constant in $u(x(t), t)$:
\begin{equation}
\begin{aligned}
u(x(t), t) = A\\
u(x(t), 0) = u_0(x_0) = A\\
u_0(x_0) = 0\\
A = 0
\end{aligned}
\end{equation}
From this it follows that when $x > 2t$:
\begin{tcolorbox}[minipage,colback=white,arc=0pt,outer arc=0pt]
\begin{equation}
u(x, t) = 0
\end{equation}
\end{tcolorbox}
\subsection*{b.)}
To solve for $u$ when $x < 2t$, I took the same approach, but integrated $\frac{dx}{dt}$ differently in order to get $t$ in terms of $x$ and some $t_0$:
\begin{equation}
\begin{aligned}
\frac{dx}{dt} = 2\\
dt = \frac{2}{dx}\\
\int dt = \int \frac{2}{dx}\\
t = \frac{1}{2}x + t_0
\end{aligned}
\end{equation}
From this equation we can see that $t_0 = t - \frac{1}{2}x$ and so when $x < 2t$, the value of $t_0$ will be greater than $0$ and thus defined by the initial condition $u_0 = u(0,t) = \frac{t}{1+t^2}$. From this, we follow the same procedure as above to solve for $u$:
\begin{equation}
\begin{aligned}
u(x(t), t) = A\\
u(0, t) = u_0(t_0)\\
u_0(t_0) = \frac{t_0}{1+t_0^2} = A
\end{aligned}
\end{equation}
Then plugging in the value above for $t_0$ we arrive at the solution for the value of $u(x, t)$ when $x < 2t$:
\begin{tcolorbox}[minipage,colback=white,arc=0pt,outer arc=0pt]
\begin{equation}
u(x, t) =  \frac{t - \frac{1}{2}x}{1+(t - \frac{1}{2}x)^2}
\end{equation}
\end{tcolorbox}
Here are the profiles of the solution at $t = 0, 1, 2, 3$.
\begin{figure}[H]
  \centering
    \includegraphics[width=\textwidth]{hw_12_plot2.pdf}
    \caption{$t = 0, 1, 2, 3$}
\end{figure}
\section*{Part 3}
This problem yields the same characteristic lines as in Part 1, namely:
\begin{equation}
\begin{aligned}
\frac{dx}{dt} = 2\\
x = 2t + x_0
\end{aligned}
\end{equation}
However, the second ODE is now:
\begin{equation}
\frac{du}{dt} = - u(x(t), t)
\end{equation}
This equation describes the rate of change of $u(x(t), t)$ along the characteristic lines, and is a function entirely of $t$, and it can be solved as follows:
\begin{equation}
\begin{aligned}
\int \frac{du}{u} = \int -dt\\
ln(u) = -t + C\\
u(x(t), t) = e^{-t + C} = e^Ce^{-t} = Ae^{-t}
\end{aligned}
\end{equation}
We now use the same method as above to solve for the constant $A$:
\begin{equation}
\begin{aligned}
u(x, 0)= u(x(0), 0) = u_0(x_0) = e^{-x_0^2}\\
u(x(t), 0) = Ae^0\\
A = e^{-x_0^2}
\end{aligned}
\end{equation}
So we see that along the characteristic lines we have some constant defined by the initial condition $x_0$, and that over the characteristic line this value is damped over time. While this could be simplified to a single exponent, I prefer to leave them separate so as to elucidate the damping term.
\begin{equation}
u(x(t), t) = e^{-t}e^{-x_0^2}
\end{equation}
Finally, we can plug in the value for $x_0$ in terms of $(x,t)$ and arrive at a general solution:
\begin{tcolorbox}[minipage,colback=white,arc=0pt,outer arc=0pt]
\begin{equation}
u(x, t) = e^{-t}e^{-(x-2t)^2}
\end{equation}
\end{tcolorbox}
To compare this with Part 1, let us calculate $(x,t)$:
\begin{equation}
u(3, 4) = e^{-4}e^{-(3-2(4))^2}= = e^{-4}e^{-25}
\end{equation}
\begin{tcolorbox}[minipage,colback=white,arc=0pt,outer arc=0pt]
\begin{equation}
u(3,4) = e^{-29}
\end{equation}
\end{tcolorbox}
As expected, effect of the $-u$ term has been to dampen the solution over time relative to the solution in Part 1.\\
\\
The request that we draw the characteristic lines in the (t, x) plane seems a little odd to me, and perhaps it is a typo? The characteristic lines for this problem are identical to those in Part 1, and (Figure 1) above displays them. If, however, it is not a typo, we can re-orient the lines in Figure 1 by determining $t$ in terms of $x$ and $t_0$ as was done in equation (14) above, then plot them:
\begin{figure}[H]
  \centering
    \includegraphics[width=\textwidth]{hw_12_plot3.pdf}
    \caption{$t = 0, 1, 2, 3$}
\end{figure}
\end{document}