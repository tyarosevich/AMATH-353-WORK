\documentclass{article}

\usepackage{siunitx} % Provides the \SI{}{} and \si{} command for typesetting SI units
\usepackage{graphicx} % Required for the inclusion of images
\usepackage{amsmath} % Required for some math elements 
\usepackage[export]{adjustbox} % loads also graphicx
\usepackage{listings}
\usepackage{matlab-prettifier}
\usepackage{float}
\usepackage[most]{tcolorbox}
\usepackage{amsfonts}

\usepackage{titlesec}
\usepackage{caption}
\usepackage{subcaption}

\newcommand{\R}{\mathbb{R}}

\usepackage{xcolor}

\DeclareCaptionFont{white}{\color{white}}
\DeclareCaptionFormat{listing}{%
  \parbox{\textwidth}{\colorbox{gray}{\parbox{\textwidth}{#1#2#3}}\vskip-4pt}}
\captionsetup[lstlisting]{format=listing,labelfont=white,textfont=white}
\lstset{frame=lrb,xleftmargin=\fboxsep,xrightmargin=-\fboxsep}
\titleformat{\section}[runin]
  {\normalfont\Large\bfseries}{\thesection}{1em}{}
\titleformat{\subsection}[runin]
  {\normalfont\large\bfseries}{\thesubsection}{1em}{}


\setlength\parindent{0pt} % Removes all indentation from paragraphs

\renewcommand{\labelenumi}{\alph{enumi}.} % Make numbering in the enumerate environment by letter rather than number (e.g. section 6)

%\usepackage{times} % Uncomment to use the Times New Roman font

%----------------------------------------------------------------------------------------
%	DOCUMENT INFORMATION
%----------------------------------------------------------------------------------------

\title{AMATH 353: Homework 8 \\Due May, 2 2018 \\ ID: 1064712} % Title

\author{Trent \textsc{Yarosevich}} % Author name

\date{\today} % Date for the report

\begin{document}
\maketitle % Insert the title, author and date
\setlength\parindent{1cm}

\begin{center}
\begin{tabular}{l r}
%Date Performed: December 1, 2017 \\ % Date the experiment was performed
Instructor: Jeremy Upsal % Instructor/supervisor
\end{tabular}
\end{center}

% If you wish to include an abstract, uncomment the lines below
% \begin{abstract}
% Abstract text
% \end{abstract}

%----------------------------------------------------------------------------------------
%	SECTION 1
%----------------------------------------------------------------------------------------
\section*{Part 1} 
\subsection*{a.)}
Assuming solutions of the form $u(x,t) = G(t)F(x)$ and the PDE $u_{tt} + u_{xx} = 0$ we get the following:
\begin{equation}
\begin{aligned}
G''(t)F(x) = -F''(x)G(t)\\
\frac{G''(t)}{G(t)} = -\frac{F''(x)}{F(x)} = \lambda
\end{aligned}
\end{equation}
The two equations are equal to the constant $\lambda $ because neither the RHS or LHS is in terms of its independent variable. This results in the two ODEs:
%\begin{tcolorbox}[minipage,colback=white,arc=0pt,outer arc=0pt]
\begin{equation}
\begin{aligned}
G''(t) = \lambda G(t)\\
F''(x) = - \lambda F(x)
\end{aligned}
\end{equation}
%\end{tcolorbox}
\subsection*{b.)}
Assuming solutions of the form $u(x,t) = G(t)F(x)$ and the PDE $u_t = \kappa u_{xx}$ we get the following:
\begin{equation}
\begin{aligned}
G'(t)F(x) = \kappa F''(x)G(t)\\
\frac{G'(t)}{G(t)} = \kappa \frac{F''(x)}{F(x)} = \lambda
\end{aligned}
\end{equation}
The two equations are equal to the constant $\lambda $ as above, yielding the two ODEs:
%\begin{tcolorbox}[minipage,colback=white,arc=0pt,outer arc=0pt]
\begin{equation}
\begin{aligned}
G'(t) = \lambda G(t)\\
F''(x) = \frac{\lambda}{\kappa} F(x)
\end{aligned}
\end{equation}
%\end{tcolorbox}
\section*{Part 2}
\subsection*{a.)}
Starting with the equation $u_{tt} + u_{xx} = 0$ we have the following $x$ dependent ODE and three cases for $\lambda$:
\begin{equation}
F''(x) = -\lambda F(x)
\end{equation}
\subsection*{$lambda = 0$}~\newline 
\\
In this case we have $F''(x) = - 0 * F(x)$. Integrating twice gives $F(x) = A + Bx$. With the BC $u(0,t) = 0$ we get $F(0) = A + B(0) = 0$ and thus $A = 0$. With the other BC $u_x(L,t) = 0$ and $F'(L) = B$ we must conclude that $B=0$ as well, yielding only trivial solutions, so \underline{$\lambda = 0$ is not an allowed value.}
\subsection*{$\lambda > 0$}~\\
\\
In this case we have $F''(x) = -\lambda F(x)$ where $\lambda$ is a positive real number. Let $\lambda = r^2$, with $r \in \mathbb{R}_{>0}$. Then by characteristic polynomial we get following, per Euler's Formula:
\begin{equation}
\begin{aligned}
F(x) = C_1e^{irx} + C_2e^{-irx}\\
F(x) = C_1\cos(rx)+C_2sin(rx)
\end{aligned}
\end{equation}
We then apply the BC $u(0,t) = 0$:
\begin{equation}
\begin{aligned}
F(0) = C_1\cos(0) + C_2\sin(0) = 0\\
C_1 = 0\\
F(x) = C_2\sin(rx)
\end{aligned}
\end{equation}
And then the second BC:
\begin{equation}
\begin{aligned}
F'(x) = C_2r\cos(rx)\\
F'(L) = C_2r\cos(rL) = 0
\end{aligned}
\end{equation}
Because $\cos(x) = 0$ at odd multiples of $\frac{\pi}{2}$ it follows that to satisfy the second BC without setting $C_2 = 0$ that we must have the argument of cosine, $rL = \frac{\pi(2n-1)}{2}$ where $n > 0$ is a positive integer. Thus with positive eigenvalues $\lambda$ we have the eigenfunctions:
\begin{equation}
F_n(x) = C_n\sin(\frac{\pi(2n-1)x}{2L} \text{ , } n \in \mathbb{Z}^+
\end{equation}
\subsection*{$\lambda < 0$}~\\
\\
In the same fashion as above, but now let $\lambda = -r^2$, $r \in \mathbb{R}_{>0}$. From characteristic polynomial we then get:
\begin{equation}
F(x) = C_1e^{rx} + C_2e^{-rx}
\end{equation}
Applying the first BC we get:
\begin{equation}
\begin{aligned}
F(0) = C_1e^{0} + C_2e^{-0} = 0\\
C_1 + C_2 = 0\\
C_2 = - C_1\\
F(x) = C(e^{rx} - e^{-rx})
\end{aligned}
\end{equation}
Then applying this to the second BC:
\begin{equation}
\begin{aligned}
F'(x) = C(re^{rx} + re^{-rx})\\
F'(L) = C(re^{rL} + re^{-rL}) = 0\\
e^{rL} = -e^{-rL}
\end{aligned}
\end{equation}
Since there is no possibility of $e^{rL} = -e^{-rL}$, this means the only way to satisfy the BCs with $\lambda < 0$ is to have $C = 0$, yielding a trivial solution. Thus \underline{$\lambda < 0$ is not an allowed value.}
\subsection*{b.)}
We now consider the same possible values for $\lambda$ for the second equation from above, and the ODE for $x$:
\begin{equation}
\begin{aligned}
u_t = \kappa u_{xx}\\
F''(x) = \frac{\lambda}{\kappa} F(x)
\end{aligned}
\end{equation}
\subsection*{$\lambda = 0$}~\\
\\
Similarly to the previous equation, setting $\lambda = 0$ and integrating twice gives us
\begin{equation}
F(x) = A + Bx
\end{equation}
which gives only trivial solutions in a manner identical to above, Part 2, subsection a.).

\end{document}