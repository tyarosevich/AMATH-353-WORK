\documentclass{article}

\usepackage{siunitx} % Provides the \SI{}{} and \si{} command for typesetting SI units
\usepackage{graphicx} % Required for the inclusion of images
\usepackage{amsmath} % Required for some math elements 
\usepackage[export]{adjustbox} % loads also graphicx
\usepackage{listings}
\usepackage{matlab-prettifier}
\usepackage{float}
\usepackage[most]{tcolorbox}
\usepackage{amsfonts}
\usepackage{color}
\usepackage{titlesec}
\usepackage{caption}
\usepackage{subcaption}
\usepackage{placeins}
\makeatletter
\renewcommand{\env@cases}[1][@{}l@{\quad}l@{}]{%
  \let\@ifnextchar\new@ifnextchar
  \left\lbrace
  \def\arraystretch{1.2}%
  \array{#1}%
}
\makeatother

\newcommand{\R}{\mathbb{R}}

\usepackage{xcolor}

\DeclareCaptionFont{white}{\color{white}}
\DeclareCaptionFormat{listing}{%
  \parbox{\textwidth}{\colorbox{gray}{\parbox{\textwidth}{#1#2#3}}\vskip-4pt}}
\captionsetup[lstlisting]{format=listing,labelfont=white,textfont=white}
\lstset{frame=lrb,xleftmargin=\fboxsep,xrightmargin=-\fboxsep}
\titleformat{\section}[runin]
  {\normalfont\Large\bfseries}{\thesection}{1em}{}
\titleformat{\subsection}[runin]
  {\normalfont\large\bfseries}{\thesubsection}{1em}{}


\setlength\parindent{0pt} % Removes all indentation from paragraphs

\renewcommand{\labelenumi}{\alph{enumi}.} % Make numbering in the enumerate environment by letter rather than number (e.g. section 6)

%\usepackage{times} % Uncomment to use the Times New Roman font

%----------------------------------------------------------------------------------------
%	DOCUMENT INFORMATION
%----------------------------------------------------------------------------------------

\title{AMATH 353: Homework 15 \\Due June, 1 2018 \\ ID: 1064712} % Title

\author{Trent \textsc{Yarosevich}} % Author name

\date{\today} % Date for the report

\begin{document}
\maketitle % Insert the title, author and date
\setlength\parindent{1cm}

\begin{center}
\begin{tabular}{l r}
%Date Performed: December 1, 2017 \\ % Date the experiment was performed
Instructor: Jeremy Upsal % Instructor/supervisor
\end{tabular}
\end{center}

% If you wish to include an abstract, uncomment the lines below
% \begin{abstract}
% Abstract text
% \end{abstract}

%----------------------------------------------------------------------------------------
%	SECTION 1
%----------------------------------------------------------------------------------------
\section*{Part 1}
First, let's find the solutions a long the characteristic lines using the method of characteristics.
\begin{equation}
\begin{aligned}
\frac{du}{dt} = 0\\
u(x(t), t) = A\\
u_0(x_0) = A\\
u_0(x_0) = 
  \begin{cases}
			u_1, \; \; \; x_0 \leq 0 \\
			0  ,  \; \; \; x_0 > 0 \\
            \end{cases}
\
\end{aligned}
\end{equation}
The characteristic lines are then given as follows:
\begin{equation}
\begin{aligned}
c(u) = v_1(1 - 2 \frac{2u}{u_1})\\
c(u_0(x_0)) = 
	\begin{cases}
		v_1(1 - 2 \frac{2u_1}{u_1}) = -v_1, \; \; \; x_0 \leq 0 \\
		v_1(1 - 2 \frac{0}{u_1}) = v_1, \; \; \; x_0 > 0 \\
		\end{cases}
\\
x(t) = 
	\begin{cases}
		x_0 -v_1t, \; \; \; x_0 \leq 0 \\
		x_0 + v_1t, \; \; \; x_0 > 0 \\
		\end{cases}
\
\end{aligned}
\end{equation}
From this we get the following values for $x_0$:
\begin{equation}
x_0 = 
	\begin{cases}
		x + v_1t \; \; \; x_0 \leq 0 \\
		x -v_1t, \; \; \; x_0 > 0 \\
		\end{cases}
\
\end{equation}
We can then plug these into the result of (1) above to get the solution to $u$ along the characteristic lines only:
\begin{equation}
u(x,t) = 
  \begin{cases}
			u_1, \; \; \; x \leq -v_1t \\
			0  ,  \; \; \; x > v_1t \\
            \end{cases}
\
\end{equation}
This leaves the rarefaction wave. We know that, taking from the notes and Knobel, the solution along the rarefaction wave will be of the form $u(x,t) = g(\frac{x}{t})$, with the rarefaction lines taking the form $x = ct$. Plugging this into the traffic problem PDE simplifies to
\begin{equation}
\frac{1}{t}g'(x/t)\Big(g(x/t) - \frac{x}{t} \Big) = 0
\end{equation}
We know that $g$ is not constant (otherwise we'd still have a discontinuity), so we can examine $g(x/t) - \frac{x}{t} = 0$. This means that $g(x/t) = \frac{x}{t}$. Making use of this, and the fact that from our PDE we know $c = v_1(1-\frac{2u}{u_1})$, we can solve for $u$ in the following manner:
\begin{equation}
\begin{aligned}
c = \frac{x}{t}\\
g(x/t) = g(c) = c = v_1(1-\frac{2u}{u_1})\\
v_1(1-\frac{2u}{u_1}) = \frac{x}{t}\\
v_1 - \frac{2uv_1}{u_1} = \frac{x}{t}\\
\frac{2uv_1}{u_1} = -\frac{x}{t} + v_1\\
2uv_1 = u_1(v_1 - \frac{x}{t}\\
u = \frac{u_1}{2}(1 - \frac{x}{v_1t})
\end{aligned}
\end{equation}
We can then insert this solution into the region that the characteristics in (2) leave undefined, and substitute the $x_0$ values into the piecewise solution to get the final answer:
\begin{tcolorbox}[minipage,colback=white,arc=0pt,outer arc=0pt]
\begin{equation}
u(x,t) = 
  \begin{cases}[@{}l@{\quad}r@{}l@{}]
			u_1, & &x \leq -v_1t  \\
			 \frac{u_1	}{2}(1 - \frac{x}{v_1t}), & -v_1t < {} & x < v_1t\\
			0  , & &x > v_1t \\
            \end{cases}
\
\end{equation}
\end{tcolorbox}
\end{document}