\documentclass{article}

\usepackage{siunitx} % Provides the \SI{}{} and \si{} command for typesetting SI units
\usepackage{graphicx} % Required for the inclusion of images
\usepackage{amsmath} % Required for some math elements 
\usepackage[export]{adjustbox} % loads also graphicx
\usepackage{listings}
\usepackage{matlab-prettifier}
\usepackage{float}
\usepackage[most]{tcolorbox}
\usepackage{amsfonts}
\usepackage{color}
\usepackage{titlesec}
\usepackage{caption}
\usepackage{subcaption}

\newcommand{\R}{\mathbb{R}}

\usepackage{xcolor}

\DeclareCaptionFont{white}{\color{white}}
\DeclareCaptionFormat{listing}{%
  \parbox{\textwidth}{\colorbox{gray}{\parbox{\textwidth}{#1#2#3}}\vskip-4pt}}
\captionsetup[lstlisting]{format=listing,labelfont=white,textfont=white}
\lstset{frame=lrb,xleftmargin=\fboxsep,xrightmargin=-\fboxsep}
\titleformat{\section}[runin]
  {\normalfont\Large\bfseries}{\thesection}{1em}{}
\titleformat{\subsection}[runin]
  {\normalfont\large\bfseries}{\thesubsection}{1em}{}


\setlength\parindent{0pt} % Removes all indentation from paragraphs

\renewcommand{\labelenumi}{\alph{enumi}.} % Make numbering in the enumerate environment by letter rather than number (e.g. section 6)

%\usepackage{times} % Uncomment to use the Times New Roman font

%----------------------------------------------------------------------------------------
%	DOCUMENT INFORMATION
%----------------------------------------------------------------------------------------

\title{AMATH 353: Homework 10 \\Due May, 8 2018 \\ ID: 1064712} % Title

\author{Trent \textsc{Yarosevich}} % Author name

\date{\today} % Date for the report

\begin{document}
\maketitle % Insert the title, author and date
\setlength\parindent{1cm}

\begin{center}
\begin{tabular}{l r}
%Date Performed: December 1, 2017 \\ % Date the experiment was performed
Instructor: Jeremy Upsal % Instructor/supervisor
\end{tabular}
\end{center}

% If you wish to include an abstract, uncomment the lines below
% \begin{abstract}
% Abstract text
% \end{abstract}

%----------------------------------------------------------------------------------------
%	SECTION 1
%----------------------------------------------------------------------------------------
\section*{Part 1}
We consider the heat equation and the following IBVP:
\begin{align*}
&u_t=4u_{xx} & 0<&x<1, & t&> 0\\
&u(0, t)= u_x(1, t) = 0 && & t&> 0\\
&u(x,0)= x(1-x). && &&
\end{align*}
I've used separation of variables in the same fashion as I did in HW 8, which is to say I put the $k$ term with the $x$ equation. Using $k = 4$ this results in:
\begin{equation}
\begin{aligned}
G'(t) = \lambda G(t)\\
F''(x) = \frac{\lambda}{4}F(x)
\end{aligned}
\end{equation}
As in HW 8, the only allowed $\lambda$ values with this setup were $\lambda < 0$, which resulted in the following using the characteristic polynomial and Euler's Formula, and $\lambda = -r^2, r \in \mathbb{R}^+$:
\begin{equation}
F(x) = C_1\cos(\frac{r}{2}x) + C_2\sin(\frac{r}{2}x)
\end{equation}
We then apply the BCs to this:
\begin{equation}
\begin{aligned}
F(0) = C_1\cos(0) + C_2\sin(0) = 0\\
C_1 = 0\\
F'(x) = \frac{r}{2}C_2\cos(\frac{r}{2}x)\\
F'(1) = \frac{r}{2}C_2\cos(\frac{r}{2}) = 0
\end{aligned}
\end{equation}
Thus without setting $C_2 = 0$ the BC is only satisfied when $\cos(\frac{r}{2}) = 0$, which means the argument is equal to odd integer multiples of $\frac{\pi}{2}$:
\begin{equation}
\begin{aligned}
n \in \mathbb{Z}^+\\
\frac{r}{2} = \frac{\pi (2n-1)}{2}\\
r = \pi (2n-1)\\
\lambda_n = - ( \pi(2n-1))^2\\
F_n(x) = C_2\sin(\frac{\pi(2n-1)}{2}x)
\end{aligned}
\end{equation}
Turning to the equation $G'(t) = \lambda G(t)$, we can simply solve with separation of variables (ODE 101 version):
\begin{equation}
\begin{aligned}
\int \frac{dG}{dt} = \int \lambda G(t)\\
ln(G(t)) = \lambda t + C\\
G(t) = Ae^{\lambda t}
\end{aligned}
\end{equation}
Combining the above with the result in equation (4) and substituting $\lambda_n$, we get a solution for $u$, and note I have consolidated the product of the arbitrary constants in a single new arbitrary constant. Consequently, we can also use superposition to rewrite the equation as a sum of solutions.
\begin{tcolorbox}[minipage,colback=white,arc=0pt,outer arc=0pt]
\begin{equation}
\begin{aligned}
u_n(x, t) = Ae^{\lambda_n t}\sin(\frac{\pi(2n-1)}{2}x)\\
u(x, t) = \sum_{n=1}^{\infty} A_ne^{\lambda_n t}\sin(\frac{\pi(2n-1)}{2}x)
\end{aligned}
\end{equation}
\end{tcolorbox}
We now have to match this form to the initial condition $u(x, 0) = x(1-x)$, which gives us the following equation. Let this IC be $f(x)$:
\begin{equation}
\begin{aligned}
u(x, 0) = \sum_{n=1}^{\infty} A_ne^{\lambda_n (0)}\sin(\frac{\pi(2n-1)}{2}x) = f(x)\\ 
f(x) = \sum_{n=1}^{\infty} A_n\sin(\frac{\pi(2n-1)}{2}x)
\end{aligned}
\end{equation}
This gives us the Fourier Series form of the initial condition. Because we need to make use of the orthogonality relations, and $f(x)$ is only defined from $x \in [0, 1]$, we will use the odd extension of $f(x)$ because the BCs specify a fixed point at the origin. And indeed, using the even extension in this situation just results in a $0$ constant anyway.\\
\\
We define the odd extension as follows:
\[f_o(x)=
  \begin{cases}
			f(x), \; \; \; x \geq 0 \\
			-f(-x), \; \; \; x < 0
            \end{cases}
\]
bla
\end{document}