\documentclass{article}

\usepackage{siunitx} % Provides the \SI{}{} and \si{} command for typesetting SI units
\usepackage{graphicx} % Required for the inclusion of images
\usepackage{amsmath} % Required for some math elements 
\usepackage[export]{adjustbox} % loads also graphicx
\usepackage{listings}
\usepackage{matlab-prettifier}
\usepackage{float}

\usepackage{titlesec}
\usepackage{caption}
\usepackage{subcaption}

\newcommand{\R}{\mathbb{R}}

\usepackage{xcolor}

\DeclareCaptionFont{white}{\color{white}}
\DeclareCaptionFormat{listing}{%
  \parbox{\textwidth}{\colorbox{gray}{\parbox{\textwidth}{#1#2#3}}\vskip-4pt}}
\captionsetup[lstlisting]{format=listing,labelfont=white,textfont=white}
\lstset{frame=lrb,xleftmargin=\fboxsep,xrightmargin=-\fboxsep}
\titleformat{\section}[runin]
  {\normalfont\Large\bfseries}{\thesection}{1em}{}
\titleformat{\subsection}[runin]
  {\normalfont\large\bfseries}{\thesubsection}{1em}{}


\setlength\parindent{0pt} % Removes all indentation from paragraphs

\renewcommand{\labelenumi}{\alph{enumi}.} % Make numbering in the enumerate environment by letter rather than number (e.g. section 6)

%\usepackage{times} % Uncomment to use the Times New Roman font

%----------------------------------------------------------------------------------------
%	DOCUMENT INFORMATION
%----------------------------------------------------------------------------------------

\title{AMATH 353: Homework 7 \\Due April, 23 2018 \\ ID: 1064712} % Title

\author{Trent \textsc{Yarosevich}} % Author name

\date{\today} % Date for the report

\begin{document}
\maketitle % Insert the title, author and date
\setlength\parindent{1cm}

\begin{center}
\begin{tabular}{l r}
%Date Performed: December 1, 2017 \\ % Date the experiment was performed
Instructor: Jeremy Upsal % Instructor/supervisor
\end{tabular}
\end{center}

% If you wish to include an abstract, uncomment the lines below
% \begin{abstract}
% Abstract text
% \end{abstract}

%----------------------------------------------------------------------------------------
%	SECTION 1
%----------------------------------------------------------------------------------------
\section*{Part 1.)} 
Since $f_0$ is an odd function, we have:
\begin{equation}
\begin{aligned}
f_0(z) = -f_0(-z)\\
f'_0(z) = f'_0(-z)\\
f_0''(z) = -f_0''(-z)
\end{aligned}
\end{equation}
First we extend the problem to the entire real line and define $f_0$ and $g_0$ as follows:
\[f_0(x) =
  \begin{cases}
               f(x) , x > 0\\
               -f(-x), x < 0
            \end{cases}
\]
and
\[g_0(x) =
  \begin{cases}
               g(x) , x > 0\\
               -g(-x), x < 0
            \end{cases}
\]
This gives
\begin{equation}
u(x,t) = \frac{1}{2}(f_0(x-ct) + f(x+ct)) + \frac{1}{2c}\int_{x-ct}^{x+ct}g_0(s)ds
\end{equation}
\\
Now the side conditions, with the integral of $g_0$ being zero because it is a definite integral across a zero domain:
\begin{equation}
u(x,0) = \frac{1}{2}(f_0(x) + f_0(x)) + 0 = f_0(x)
\end{equation}
And since solutions are restricted to $x\geq 0$ and there is no $t$ term we have $f_0(x) = f(x)$.\\
\\
Now consider the following with $G_0(s) = \int g_0(s)ds$:
\begin{equation}
\begin{aligned}
u_t(x, t) = \frac{1}{2}(-cf_{0,t}(x-ct) + cf_{0,t}(x+ct)) + \frac{1}{2c} \frac{\partial}{\partial t}(G_0(x+ct) - G_0(x - ct))\\
u_t(x, t) = \frac{1}{2}(-cf_{0,t}(x-ct) + cf_{0,t}(x+ct)) + \frac{1}{2c}(cg_0(x+ct) + cg_0(x - ct))\\
u_t(x, 0) = \frac{1}{2}(-cf_0'(x) + cf_0'(x)) + \frac{1}{2c}(cg_0(x) + cg_0(x))\\
u_t(x,0) = 0 + g_0(x)
\end{aligned}
\end{equation}
And again because of the restricted $x$ and $t = 0$ we get $u_t(x,0) = g_0(x) = g(x)$.\\
\\
Lastly for the BC we have the following:
\begin{equation}
u(0, t) = \frac{1}{2}(f_0(-ct) + f_0(ct)) + \frac{1}{2c}(G_0(ct) - G_0(-ct))\\
\end{equation}
And because $f_0$ is an odd function, $f_0(-ct) = -f_0(ct)$, and because $G_0$ is the integral of an odd function, I am assuming it must be an even function\footnote{I didn't look all that hard for an authoritative proof of this, but there is one here that my admittedly ignorant self finds convincing: https://math.stackexchange.com/questions/2227677/proof-that-integral-of-odd-function-is-even-function}, so $G_0(-ct) = G_0(ct)$ yielding:
\begin{equation}
u(0, t) = \frac{1}{2}(-f_0(ct) + f_0(ct)) + \frac{1}{2c}(G_0(ct) - G_0(ct))= 0\\
\end{equation}
\section*{Part 2}
\[f_0(x) =
  \begin{cases}
               -\cos (x) + 1 , x > 0\\
               \cos(x) - 1, x < 0
            \end{cases}
\]
and
\[g_0(x) =
  \begin{cases}
               xe^{-x^2} , x > 0\\
               xe^{-x^2}, x < 0
            \end{cases}
\]
Note that $xe^{-x^2}$ is the odd extension of itself and $\int g_0(s)ds = -\frac{1}{2}e^{-(s)^2}$ (constant of integration is zero), and $c = \sqrt{81} = 9$. From this we get the solution:
\begin{equation}
u(x,t) = \frac{1}{2}(f_0(x - 9t) + f_0(x + 9t))  +\frac{1}{36}(-e^{-(x+9t)^2} + e^{-(x-9t)^2})
\end{equation}
\end{document}