\documentclass{article}

\usepackage{siunitx} % Provides the \SI{}{} and \si{} command for typesetting SI units
\usepackage{graphicx} % Required for the inclusion of images
\usepackage{amsmath} % Required for some math elements 
\usepackage[export]{adjustbox} % loads also graphicx
\usepackage{listings}
\usepackage{matlab-prettifier}
\usepackage{float}
\usepackage[most]{tcolorbox}
\usepackage{amsfonts}

\usepackage{titlesec}
\usepackage{caption}
\usepackage{subcaption}

\newcommand{\R}{\mathbb{R}}

\usepackage{xcolor}

\DeclareCaptionFont{white}{\color{white}}
\DeclareCaptionFormat{listing}{%
  \parbox{\textwidth}{\colorbox{gray}{\parbox{\textwidth}{#1#2#3}}\vskip-4pt}}
\captionsetup[lstlisting]{format=listing,labelfont=white,textfont=white}
\lstset{frame=lrb,xleftmargin=\fboxsep,xrightmargin=-\fboxsep}
\titleformat{\section}[runin]
  {\normalfont\Large\bfseries}{\thesection}{1em}{}
\titleformat{\subsection}[runin]
  {\normalfont\large\bfseries}{\thesubsection}{1em}{}


\setlength\parindent{0pt} % Removes all indentation from paragraphs

\renewcommand{\labelenumi}{\alph{enumi}.} % Make numbering in the enumerate environment by letter rather than number (e.g. section 6)

%\usepackage{times} % Uncomment to use the Times New Roman font

%----------------------------------------------------------------------------------------
%	DOCUMENT INFORMATION
%----------------------------------------------------------------------------------------

\title{AMATH 353: Homework 8 \\Due May, 2 2018 \\ ID: 1064712} % Title

\author{Trent \textsc{Yarosevich}} % Author name

\date{\today} % Date for the report

\begin{document}
\maketitle % Insert the title, author and date
\setlength\parindent{1cm}

\begin{center}
\begin{tabular}{l r}
%Date Performed: December 1, 2017 \\ % Date the experiment was performed
Instructor: Jeremy Upsal % Instructor/supervisor
\end{tabular}
\end{center}

% If you wish to include an abstract, uncomment the lines below
% \begin{abstract}
% Abstract text
% \end{abstract}

%----------------------------------------------------------------------------------------
%	SECTION 1
%----------------------------------------------------------------------------------------
\section*{Part 1}
We are asked to compute the integral $\int_a^b x\cos(\frac{n\pi x}{2})$. I will also show my work here for $\int_a^b (2-x)\cos(\frac{n\pi x}{2})$ as this is how I computed the $a_n$ term later on. Starting with the first one, using division by parts we have:
\begin{equation}
\begin{aligned}
\int x\cos(\frac{n\pi x}{2}) = \int udv\\
u = x \; , \; du = dx\\
dv = \cos(\frac{n\pi x}{2})\\
v = \int dv = \frac{2}{n\pi}\sin(\frac{n\pi x}{2})\\
\int udv = uv - \int vdu = \frac{2x}{n\pi}\sin(\frac{n\pi x}{2}) + \frac{4}{n^2\pi^2}\cos(\frac{n\pi x}{2})
\end{aligned}
\end{equation}
\begin{multline}
\int_a^b x\cos(\frac{n\pi x}{2}) = \\ \frac{2b}{n\pi}\sin(\frac{n\pi b}{2}) + \frac{4}{n^2\pi^2}\cos(\frac{n\pi b}{2}) - \frac{2a}{n\pi}\sin(\frac{n\pi a}{2}) - \frac{4}{n^2\pi^2}\cos(\frac{n\pi a}{2})
\end{multline}
And now the same for $\int_a^b (2-x)\cos(\frac{n\pi x}{2})$. Note that I did the definite integral directly here, unlike the previous equation.
\begin{equation}
\begin{aligned}

\end{aligned}
\end{equation}
\end{document}