\documentclass{article}

\usepackage{siunitx} % Provides the \SI{}{} and \si{} command for typesetting SI units
\usepackage{graphicx} % Required for the inclusion of images
\usepackage{amsmath} % Required for some math elements 
\usepackage[export]{adjustbox} % loads also graphicx
\usepackage{listings}
\usepackage{matlab-prettifier}
\usepackage{float}

\usepackage{titlesec}
\usepackage{caption}
\usepackage{subcaption}


\usepackage{xcolor}

\DeclareCaptionFont{white}{\color{white}}
\DeclareCaptionFormat{listing}{%
  \parbox{\textwidth}{\colorbox{gray}{\parbox{\textwidth}{#1#2#3}}\vskip-4pt}}
\captionsetup[lstlisting]{format=listing,labelfont=white,textfont=white}
\lstset{frame=lrb,xleftmargin=\fboxsep,xrightmargin=-\fboxsep}
\titleformat{\section}[runin]
  {\normalfont\Large\bfseries}{\thesection}{1em}{}
\titleformat{\subsection}[runin]
  {\normalfont\large\bfseries}{\thesubsection}{1em}{}


\setlength\parindent{0pt} % Removes all indentation from paragraphs

\renewcommand{\labelenumi}{\alph{enumi}.} % Make numbering in the enumerate environment by letter rather than number (e.g. section 6)

%\usepackage{times} % Uncomment to use the Times New Roman font

%----------------------------------------------------------------------------------------
%	DOCUMENT INFORMATION
%----------------------------------------------------------------------------------------

\title{AMATH 353: Homework 2 \\Due April, 6 2018 \\ ID: 1064712} % Title

\author{Trent \textsc{Yarosevich}} % Author name

\date{\today} % Date for the report

\begin{document}
\maketitle % Insert the title, author and date
\setlength\parindent{1cm}

\begin{center}
\begin{tabular}{l r}
%Date Performed: December 1, 2017 \\ % Date the experiment was performed
Instructor: Jeremy Uscal % Instructor/supervisor
\end{tabular}
\end{center}

% If you wish to include an abstract, uncomment the lines below
% \begin{abstract}
% Abstract text
% \end{abstract}

%----------------------------------------------------------------------------------------
%	SECTION 1
%----------------------------------------------------------------------------------------
\section*{} 
	In order to change the range of $t$ to some $\tau > 0$, I used the following function: 
\begin{equation}
\tau(t) = t - T
\end{equation}

\noindent By plugging this new variable into $u$ and using the chain rule we get: 
\begin{equation}
\begin{aligned}
\hat{u}_{tt} = \frac{d}{dt}(\frac{du}{dt})\\
\hat{u}_{tt} = \frac{d}{dt}(\frac{du}{d\tau}\frac{d\tau}{dt})
\end{aligned}
\end{equation}
Because $\frac{d\tau}{dt} = 1$ we then get:
\begin{equation}
\hat{u}_{tt} = \frac{d}{dt}(\frac{du}{d\tau})
\end{equation}
We then do the same procedure again:
\begin{equation}
\begin{aligned}
\hat{u}_{tt} = \frac{d^2u}{d\tau}\frac{d\tau}{dt}\\
\hat{u}_{tt} = \frac{d^2u}{d\tau} = u_{\tau\tau}
\end{aligned}
\end{equation}
The fact that there is no change in the PDE is to be expected since we are merely re-locating the function to the left, with no alterations to its rate of change. The values for the boundary conditions are adjusted accordingly to $\hat{u}(a, h_1(\tau)) = u(a, t - T)$ and $\hat{u}(b, h_2(\tau)) = u(b, t - T)$.
\end{document}