\documentclass{article}

\usepackage{siunitx} % Provides the \SI{}{} and \si{} command for typesetting SI units
\usepackage{graphicx} % Required for the inclusion of images
\usepackage{amsmath} % Required for some math elements 
\usepackage[export]{adjustbox} % loads also graphicx
\usepackage{listings}
\usepackage{matlab-prettifier}
\usepackage{float}

\usepackage{titlesec}
\usepackage{caption}
\usepackage{subcaption}


\usepackage{xcolor}

\DeclareCaptionFont{white}{\color{white}}
\DeclareCaptionFormat{listing}{%
  \parbox{\textwidth}{\colorbox{gray}{\parbox{\textwidth}{#1#2#3}}\vskip-4pt}}
\captionsetup[lstlisting]{format=listing,labelfont=white,textfont=white}
\lstset{frame=lrb,xleftmargin=\fboxsep,xrightmargin=-\fboxsep}
\titleformat{\section}[runin]
  {\normalfont\Large\bfseries}{\thesection}{1em}{}
\titleformat{\subsection}[runin]
  {\normalfont\large\bfseries}{\thesubsection}{1em}{}


\setlength\parindent{0pt} % Removes all indentation from paragraphs

\renewcommand{\labelenumi}{\alph{enumi}.} % Make numbering in the enumerate environment by letter rather than number (e.g. section 6)

%\usepackage{times} % Uncomment to use the Times New Roman font

%----------------------------------------------------------------------------------------
%	DOCUMENT INFORMATION
%----------------------------------------------------------------------------------------

\title{AMATH 353: Homework 4 \\Due April, 13 2018 \\ ID: 1064712} % Title

\author{Trent \textsc{Yarosevich}} % Author name

\date{\today} % Date for the report

\begin{document}
\maketitle % Insert the title, author and date
\setlength\parindent{1cm}

\begin{center}
\begin{tabular}{l r}
%Date Performed: December 1, 2017 \\ % Date the experiment was performed
Instructor: Jeremy Upsal % Instructor/supervisor
\end{tabular}
\end{center}

% If you wish to include an abstract, uncomment the lines below
% \begin{abstract}
% Abstract text
% \end{abstract}

%----------------------------------------------------------------------------------------
%	SECTION 1
%----------------------------------------------------------------------------------------
\section*{Part 1.)} 
With some slight rearrangement we have:
\begin{equation}
u(x,t) = \frac{3}{1 + \cos^2(7(x -\frac{5}{7}t) +2)}
\end{equation}
This is a traveling wave solution with $c = \frac{5}{7}$ and $f$, suppose some $f(z)$, as follows:
\begin{equation}
f(z) = \frac{3}{1 + \cos^2(7z +2)}
\end{equation}
\section*{Part 2.)}
\subsection*{a.)}
From linearized Burgers: $u_t + au_x = du_{xx}$ we get
\begin{equation}
\begin{aligned}
-iwu + aiku = d(ik)^2u\\
-iw + ai = d(ik)^2\\
-w + ak = dik^2\\
w = ak-dik^2\\
\end{aligned}
\end{equation}
a.) is FALSE, this $c_p(k)$ is not real for real $k$ values.\\
b.) is FALSE because the dispersion relation IS constant in respect to k, i.e. $a-dik$.
\subsection*{b.)}
From Schr\"odinger's equation: $iu_t + u_{xx} = 0$ we get
\begin{equation}
\begin{aligned}
i(-iw)u + (ik)^2u = 0\\
i(-iw) + (ik)^2 = 0\\
-i^2w + i^2k^2 = 0\\
w = k^2
\end{aligned}
\end{equation}
a.) is TRUE because there are no complex numbers in the dispersion relation (and thus the phase velocity).\\
b.) is FALSE because $c_p(k) = \frac{w}{k} = k$, so the dispersion relation IS constant with respect to k.
\subsection*{c.)}
From the wave equation $u_{tt} = au_{xx}$ we get:
\begin{equation}
\begin{aligned}
(-iw)^2u = a(ik)^2u\\
-(iw)^2 = a(ik)^2\\
-i^2w^2 = ai^2k^2\\
w^2 = -ak^2\\
w = \sqrt{-ak^2}\\
w = \pm i\sqrt{a}k
\end{aligned}
\end{equation}
a.) is FALSE because the dispersion relation (and phase velocity) contains a complex number.\\
b.) is TRUE because $c_p(k) = \pm i\sqrt{a}$, and is not dependent on $k$, so the equation is non-dispersive. Question: the equation would not be dispersive anyway because of the complex term in the dispersion relation, correct?
\end{document}